\subsection{Dataset}
Our dataset consists of 9000 sentences collected from social media comments. Every sentence has been labelled with a sentiment value. These values range between -2 and 2, where negative sentences have negative values, positive sentences have positive values and neutral sentences have a value of 0.
The sentiment values were decided according to our ruleset (APPENDIX)
About half of our sentences have been collected from Facebook, while the other half were collected from Reddit and Twitter. On Facebook and Reddit, we only collected top level comments, since they were the only ones that were almost guaranteed to be directed at the article. Since Twitter organizes replies differently than Facebook and Reddit, we collected every reply in the thread.
While gathering data for the dataset, we became aware that most of the sentences we had collected were negative or neutral. The final dataset consists of 45\% neutral sentences, 39\% negative sentences and 16\% positive sentences. 

\subsection{Determining sentiment}
We have implemented a variety of different methods of determining the sentiment of a sentence. Overall these can be separated into two main approaches: Lexical analysis and Machine learning. 

We decided to use the following three classes, when classifying sentiment: Positive (1), Neutral (0) and Negative (-1). This was mainly because we wanted to focus on the overall polarity of the sentences rather than the specifics. Another reason, was that our classifiers were more accurate when when we used three classes than when we used five.
We also did some experiments with only two classes; positive and negative. Here we were able to get get a better accuracy for our predictions,but we lost the ability to classify sentences without sentiment. Therefore we decided to stay with the three classes.

The classifiers using a machine learning approach all share the fact that they are trained on a set of data (our training set) in order to “learn” which sentences belong to which classes. This knowledge is then applied in order to guess what the sentiment of an unseen sentence would be. 
