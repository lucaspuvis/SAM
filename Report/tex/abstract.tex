\begin{abstract}
	The authors of this thesis seeks to implement a general purpose program that uses different approaches to programmably predict the sentiment of comments on Danish social media - more specifically political comment sections - in what is called Natural Language Processesing. Furthermore, to support this approach a dataset consisting of a large quantity of comments needed to be collected as well as a comprehensive amount of annotating needed to be done, such that it would be possible to evaluate the accuracy of the implementations. During the process the authors realized that the dataset was weighted towards containing more negative sentiment rather than positive, and that this trend especially was prominent on Facebook. The authors believes that there may have been flaw in the methodology that could have negatively affected the results. The thesis concludes that with significantly more data, the accuracy should increase together with widening the scope of the dataset such that it is no longer too specific.
\end{abstract}
